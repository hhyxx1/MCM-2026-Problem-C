%%%%%%%%%%%%%%%%%%%%%%%%%%%%%%%%%%%%%%%%
%% MCM/ICM LaTeX Template %%
%% 2026 MCM/ICM           %%
%%%%%%%%%%%%%%%%%%%%%%%%%%%%%%%%%%%%%%%%
\documentclass[12pt]{article}
\usepackage{geometry}
\geometry{left=1in,right=0.75in,top=1in,bottom=1in}

%%%%%%%%%%%%%%%%%%%%%%%%%%%%%%%%%%%%%%%%
% Replace ABCDEF in the next line with your chosen problem
\newcommand{\Problem}{C}
\newcommand{\Team}{2627699}
%%%%%%%%%%%%%%%%%%%%%%%%%%%%%%%%%%%%%%%%

\usepackage{newtxtext}
\usepackage{amsmath,amssymb,amsthm}
\usepackage{newtxmath} % must come after amsXXX

\usepackage[pdftex]{graphicx}
\usepackage{xcolor}
\usepackage{fancyhdr}
\usepackage{float}
\usepackage{booktabs}
\usepackage{titlesec}
\usepackage{hyperref}

\hypersetup{
    colorlinks=true,
    linkcolor=black,
    filecolor=magenta,      
    urlcolor=cyan,
    citecolor=black
}

\lhead{Team \Team}
\rhead{}
\cfoot{\thepage}

\newtheorem{theorem}{Theorem}
\newtheorem{corollary}[theorem]{Corollary}
\newtheorem{lemma}[theorem]{Lemma}
\newtheorem{definition}{Definition}

%%%%%%%%%%%%%%%%%%%%%%%%%%%%%%%%
\begin{document}
\graphicspath{{.}}  % Place your graphic files in the same directory as your main document
\DeclareGraphicsExtensions{.pdf, .jpg, .tif, .png}
\thispagestyle{empty}
\vspace*{-16ex}
\centerline{\begin{tabular}{*3{c}}
	\parbox[t]{0.3\linewidth}{\begin{center}\textbf{Problem Chosen}\\ \Large \textcolor{red}{\Problem}\end{center}}
	& \parbox[t]{0.3\linewidth}{\begin{center}\textbf{2026\\ MCM/ICM\\ Summary Sheet}\end{center}}
	& \parbox[t]{0.3\linewidth}{\begin{center}\textbf{Team Control Number}\\ \Large \textcolor{red}{\Team}\end{center}}	\\
	\hline
\end{tabular}}
%%%%%%%%%%% Begin Summary %%%%%%%%%%%
\begin{center}
\textbf{\Large Title}
\end{center}

Summary
%%%%%%%%%%% End Summary %%%%%%%%%%%

%%%%%%%%%%%%%%%%%%%%%%%%%%%%%%
\clearpage
\pagestyle{fancy}
\tableofcontents
\newpage
\setcounter{page}{1}
\rhead{Page \thepage\ }
%%%%%%%%%%%%%%%%%%%%%%%%%%%%%%

\section{Introduction}

\subsection{Background}


\subsection{Problem Restatement}


\subsection{Our Work}

\begin{figure}[H]
    \centering
    \includegraphics[width=0.9\linewidth]{cleaned_outputs/workflow_flowchart.png}
    \caption{The overall workflow of our analysis pipeline, showing the transition from data archaeology to policy recommendation.}
    \label{fig:workflow}
\end{figure}

\section{Assumptions and Notations}

\subsection{Assumptions}


\subsection{Notations}


\section{Data Archaeology and Exploratory Analysis}

\subsection{Data Preprocessing}
% 数据标准化:30分制/40分制统一为百分比
% 退赛处理:剔除N/A和0分数据
% 缺失值和异常值处理

\subsection{Feature Engineering}
% PBI (人气偏差指数) = Rank_Judge - Rank_Final
% 协变量提取:Age, Industry, Pro Partner, Season/Week

\subsection{Divergence Trend Analysis}
% 时序热力图:评委-粉丝分歧随时间变化
% 社交媒体时代(S15后)分歧显著增加
% 改革必要性论证

\begin{figure}[H]
    \centering
    \includegraphics[width=1.0\linewidth]{cleaned_outputs/global_scan/divergence_trend.png}
    \caption{Chronological Trend of Judge-Audience Divergence (S1-S34). The shaded area represents the widening gap between professional evaluation and public popularity in the social media era.}
    \label{fig:divergence}
\end{figure}

\section{Bayesian Inverse Inference Model for Fan Vote Estimation}

\subsection{Problem Formulation}
% 隐变量:f(i,w) = 粉丝票份额
% 单纯形约束:sum f = 1, f >= 0
% 淘汰约束:Score(survivor) > Score(eliminated)

\subsection{Hit-and-Run MCMC Algorithm}
% LP初始化
% 方向采样
% 线搜索
% 特殊周次处理:多人淘汰周、累积投票周

\subsection{Model Validation: Certainty and Consistency}
% 确定性:95% CI宽度,avg CIW = 0.288
% 一致性:P(E in Bottom-k | posterior)
% 全局一致性 = 65.1%,非异常周 = 95.2%

\begin{figure}[H]
    \centering
    \includegraphics[width=0.8\linewidth]{cleaned_outputs/bayesian_inference/ci_width_distribution.png}
    \caption{Distribution of 95\% Credible Interval Widths for Fan Vote Estimates. The narrow peak indicates high certainty for most observations.}
    \label{fig:ci_width}
\end{figure}

\section{Pareto Optimization Model for Dynamic Weighting Rules}

\subsection{Dual Objective Definition}
% J = Spearman(最终排名, 评委排名) - 精英选拔
% F = Spearman(最终排名, 粉丝排名) - 粉丝参与
% Balance = 2JF/(J+F) - 调和平均

\subsection{Multi-Phase Evaluation Framework}
% 阶段划分:早期/中期/后期
% 动态模式得分:(F_early - F_late) + (J_late - J_early)
% 阶段Balance:早期侧重F,后期侧重J
% 综合得分:多维度加权

\subsection{Rule Space Search}
% 静态规则:Rank制 vs Pct制
% 旧动态规则:线性 + 对数平滑
% 新动态规则:Sigmoid + Rank制

\subsection{Optimal Rule Selection}
% Pareto前沿分析
% 最优配置:Sigmoid(0.30, 0.75, 6)
% 综合得分 +21.6%,5:3胜出

\section{Rule Simulation and Mechanism Comparison}

\subsection{Simulator Architecture}
% 旧架构:Pct + Linear
% 新架构:Rank + Sigmoid
% 权重计算公式

\subsection{Rank vs. Percentage System Comparison}
% 指标:FFI, JFI, Fan-Elasticity, 周次差异
% 结果:Pct更偏向粉丝,Rank更平衡
% 定量:F_early +52.7%, J_late +67.5%

\begin{figure}[H]
    \centering
    \includegraphics[width=0.9\linewidth]{cleaned_outputs/patch4_elasticity/elasticity_comparison.png}
    \caption{Fan-Elasticity Comparison: Rank vs. Percentage System. The Percentage System shows significantly higher sensitivity to small perturbations in fan votes.}
    \label{fig:elasticity}
\end{figure}

\subsection{Historical Case Studies}
% Jerry Rice (S2): 评委低分但存活多周
% Billy Ray Cyrus (S4): 粉丝票高但最终第5
% Bristol Palin (S11): 争议性进入前3
% Bobby Bones (S27): 冠军但评委分最低
% 新规则可纠正所有4个争议结果

\section{Covariate Effect Analysis}

\subsection{Pro Dancer Effect Model}
% 评委分数模型:J% ~ Age + Industry + Pro_Partner + Week
% 粉丝票模型:logit(f) ~ Age + Industry + Pro_Partner + J%
% 方差分解:Pro Dancer随机效应解释的方差百分比

\subsection{Celebrity Industry Effect}
% 行业分类:运动员/演员/歌手/政客等
% 名人效应对评委分和粉丝票的差异化影响

\section{Model Evaluation and Sensitivity Analysis}

\subsection{Sensitivity Analysis}
% 参数敏感性:w_min, w_max, steepness
% 鲁棒性检验

\subsection{Strengths and Weaknesses}

\textbf{Strengths:}


\textbf{Weaknesses:}


\section{Conclusion}


\section{Memo to the Producer}

\subsection{Problem Diagnosis}
% 社交媒体时代刷票问题

\subsection{Recommended Solution}
% Sigmoid动态权重 + Rank制
% 公式:Score(t) = w_J(t) * J_rank + (1-w_J(t)) * F_rank
% 权重演化:30% → 75%

\begin{figure}[H]
    \centering
    \includegraphics[width=0.9\linewidth]{cleaned_outputs/phase5_recommendation/dynamic_weights.png}
    \caption{Proposed Sigmoid Dynamic Weighting Scheme. The weight of Meritocracy increases smoothly from 30\% to 75\% as the season progresses.}
    \label{fig:weights}
\end{figure}

\subsection{Expected Benefits}
% 早期粉丝参与 +52.7%
% 后期精英选拔 +67.5%
% 综合得分 +21.6%

\subsection{Optional Mechanism: Judges' Save}
% 仅对Bottom 2生效
% 评委可挑选一人进行保护

\end{document}

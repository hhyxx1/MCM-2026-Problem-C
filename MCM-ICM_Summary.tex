%%%%%%%%%%%%%%%%%%%%%%%%%%%%%%%%%%%%%%%%
%% MCM/ICM LaTeX Template %%
%% 2026 MCM/ICM           %%
%%%%%%%%%%%%%%%%%%%%%%%%%%%%%%%%%%%%%%%%
\documentclass[12pt]{article}
\usepackage{geometry}
\geometry{left=1in,right=0.75in,top=1in,bottom=1in}

%%%%%%%%%%%%%%%%%%%%%%%%%%%%%%%%%%%%%%%%
% Replace ABCDEF in the next line with your chosen problem
\newcommand{\Problem}{C}
\newcommand{\Team}{2627699}
%%%%%%%%%%%%%%%%%%%%%%%%%%%%%%%%%%%%%%%%

\usepackage{newtxtext}
\usepackage{amsmath,amssymb,amsthm}
\usepackage{newtxmath} % must come after amsXXX

\usepackage[pdftex]{graphicx}
\usepackage{xcolor}
\usepackage{fancyhdr}
\usepackage{float}
\usepackage{booktabs}
\usepackage{titlesec}
\usepackage{hyperref}

\hypersetup{
    colorlinks=true,
    linkcolor=black,
    filecolor=magenta,      
    urlcolor=cyan,
    citecolor=black
}

\lhead{Team \Team}
\rhead{}
\cfoot{\thepage}

\newtheorem{theorem}{Theorem}
\newtheorem{corollary}[theorem]{Corollary}
\newtheorem{lemma}[theorem]{Lemma}
\newtheorem{definition}{Definition}

%%%%%%%%%%%%%%%%%%%%%%%%%%%%%%%%
\begin{document}
\graphicspath{{.}}  % Place your graphic files in the same directory as your main document
\DeclareGraphicsExtensions{.pdf, .jpg, .tif, .png}
\thispagestyle{empty}
\vspace*{-16ex}
\centerline{\begin{tabular}{*3{c}}
	\parbox[t]{0.3\linewidth}{\begin{center}\textbf{Problem Chosen}\\ \Large \textcolor{red}{\Problem}\end{center}}
	& \parbox[t]{0.3\linewidth}{\begin{center}\textbf{2026\\ MCM/ICM\\ Summary Sheet}\end{center}}
	& \parbox[t]{0.3\linewidth}{\begin{center}\textbf{Team Control Number}\\ \Large \textcolor{red}{\Team}\end{center}}	\\
	\hline
\end{tabular}}
%%%%%%%%%%% Begin Summary %%%%%%%%%%%
\begin{center}
\textbf{\Large The Fairness-Engagement Equilibrium Model for DWTS Voting Reform}
\end{center}

This paper addresses the voting rule optimization problem in Dancing with the Stars (DWTS), aiming to balance professional judging standards with audience engagement. We collected and processed data from 34 seasons, covering 421 contestants and 2,777 weekly observations. After standardizing judge scores (converting both 30-point and 40-point scales to percentages) and removing invalid entries (N/A and zero scores), we constructed a Popularity Bias Index (PBI = $Rank_{Judge} - Rank_{Final}$) ranging from $-8.5$ to $+6.0$, where positive values indicate fan-driven survivals.

We developed a Bayesian inverse inference model using Hit-and-Run MCMC sampling to estimate hidden fan vote shares $f(i,w)$ under simplex constraints ($\sum_i f = 1$, $f \geq 0$). The model achieved 95.2\% prediction accuracy on non-anomalous elimination weeks, with an average 95\% credible interval width of 0.288, demonstrating reliable uncertainty quantification. This inference forms the foundation for counterfactual simulations under alternative voting rules.

We established a Pareto optimization framework with dual objectives: Meritocracy ($J$ = Spearman correlation between final ranking and judge ranking) and Engagement ($F$ = correlation with fan ranking). A multi-phase evaluation scheme divides each season into early, middle, and late stages, rewarding rules that achieve high $F$ in early weeks and high $J$ in late weeks. Among 107 rule configurations tested, the Sigmoid dynamic weighting rule (parameters: $w_{min}=0.30$, $w_{max}=0.75$, steepness$=6$) achieved the highest composite score of 0.570, outperforming the best static rule (Rank 50-50, score 0.469) by 21.6\%.

Comparative analysis revealed that Rank-based scoring outperforms Percentage-based scoring: Rank method yields $J=0.665$ versus Pct method $J=0.454$ (46.4\% improvement), while maintaining comparable engagement ($F=0.704$ vs $0.706$). Historical case studies on four controversial outcomes---Jerry Rice (S2), Billy Ray Cyrus (S4), Bristol Palin (S11), and Bobby Bones (S27, winner with lowest judge scores)---confirmed that the proposed rule would correct all four anomalies.

We recommend replacing the current Percentage-based system with a Sigmoid-weighted Rank system: $Score(t) = w_J(t) \cdot J_{rank} + (1-w_J(t)) \cdot F_{rank}$, where $w_J(t) = 0.30 + 0.45/(1+e^{-6(t/T-0.5)})$. This design increases early-stage fan engagement by 52.7\% ($F_{early}$: 0.58 $\rightarrow$ 0.89) and late-stage meritocracy by 67.5\% ($J_{late}$: 0.55 $\rightarrow$ 0.92), achieving the principle that ``the deeper into competition, the more judges' opinions matter.''
%%%%%%%%%%% End Summary %%%%%%%%%%%

%%%%%%%%%%%%%%%%%%%%%%%%%%%%%%
\clearpage
\pagestyle{fancy}
\tableofcontents
\newpage
\setcounter{page}{1}
\rhead{Page \thepage\ }
%%%%%%%%%%%%%%%%%%%%%%%%%%%%%%

\section{Introduction}

\subsection{Background}


\subsection{Problem Restatement}


\subsection{Our Work}

\begin{figure}[H]
    \centering
    \includegraphics[width=0.9\linewidth]{cleaned_outputs/workflow_flowchart.png}
    \caption{The overall workflow of our analysis pipeline, showing the transition from data archaeology to policy recommendation.}
    \label{fig:workflow}
\end{figure}

\section{Assumptions and Notations}

\subsection{Assumptions}


\subsection{Notations}


\section{Data Archaeology and Exploratory Analysis}

\subsection{Data Preprocessing}

The raw dataset contains 421 contestants across 34 seasons, with judge scores recorded in a wide format (44 score columns for weeks 1--11 $\times$ 4 judges). We identified three major data quality issues requiring preprocessing. \textbf{First}, the scoring system varied across seasons: Seasons 1--10, 13--14, 16, 27, and 29 used a 3-judge system (maximum 30 points), while the remaining 20 seasons used a 4-judge system (maximum 40 points). To ensure comparability, we normalized all scores to a percentage scale $J\% = (\text{actual score} / \text{max possible}) \times 100$. \textbf{Second}, the dataset contained 8,316 \texttt{N/A} entries (judge 4 scores in 3-judge seasons) and 6,431 zero-score entries (post-elimination weeks where contestants no longer competed). We excluded these invalid observations to ensure accurate analysis. \textbf{Third}, we transformed the wide-format data into a long-format panel structure $(i, w)$, where each row represents one contestant in one week, yielding 2,777 valid observations. Additionally, we standardized 21 raw industry categories into 9 major groups (e.g., merging ``Actor/Actress'' into ``Actor'', ``Racing Driver'' into ``Athlete''), created a binary US/Non-US region indicator, and binned contestant ages into five intervals (18--25, 26--35, 36--45, 46--55, 55+). The cleaned dataset maintains complete coverage of all 34 seasons with a mean judge score of 74.8\% (SD = 11.2\%), ready for subsequent Bayesian inference and Pareto optimization modeling.

\subsection{Feature Engineering}

To quantify the divergence between professional evaluation and audience preference, we constructed the \textbf{Popularity Bias Index (PBI)} as the core feature. For each contestant $i$, we first computed their average weekly judge ranking $\bar{R}^J_i$ across all weeks they competed, then compared it with their final placement $R^*_i$:
\begin{equation}
    \text{PBI}_i = \bar{R}^J_i - R^*_i
\end{equation}
A positive PBI indicates a ``fan favorite'' who placed better than judges predicted (e.g., Kelly Monaco in S1 with PBI = $+2.17$), while a negative PBI indicates a ``judge favorite'' who placed worse than their scores deserved (e.g., Rachel Hunter in S1 with PBI = $-2.50$). Across all 421 contestants, PBI ranged from $-8.5$ to $+6.0$ with a mean of $-0.88$ (SD = 2.14), suggesting a slight overall bias toward judge-preferred outcomes under the current rules.

We also extracted covariates for subsequent modeling. \textbf{First}, we calculated partner-level statistics by aggregating PBI for each professional dancer across their career. Dancers with consistently positive average PBI (e.g., Daniella Karagach: $+1.71$, Lacey Schwimmer: $+0.61$) were identified as ``Star Makers'' who help boost celebrity popularity, while those with negative average PBI (e.g., Derek Hough: $-0.05$, Louis van Amstel: $-0.92$) tend to partner with judge favorites. \textbf{Second}, we prepared contestant-level features including age (continuous and binned), industry category (9 groups), and region (US/Non-US) for mixed-effects modeling. \textbf{Third}, we created season and week fixed effects (34 season dummies, 11 week dummies) to control for time-varying rule changes and competition structure.

\subsection{Divergence Trend Analysis}
% 时序热力图:评委-粉丝分歧随时间变化
% 社交媒体时代(S15后)分歧显著增加
% 改革必要性论证

\begin{figure}[H]
    \centering
    \includegraphics[width=1.0\linewidth]{cleaned_outputs/global_scan/divergence_trend.png}
    \caption{Chronological Trend of Judge-Audience Divergence (S1-S34). The shaded area represents the widening gap between professional evaluation and public popularity in the social media era.}
    \label{fig:divergence}
\end{figure}

\section{Bayesian Inverse Inference Model for Fan Vote Estimation}

Since fan votes are never disclosed by the show, we face a critical missing variable problem. However, elimination outcomes contain implicit information: eliminated contestants must have the lowest combined scores. We treat fan vote shares as latent variables and employ Bayesian inference to reconstruct their posterior distribution.

\subsection{Problem Formulation}

Let $f_{i,t}$ denote the proportion of fan votes received by contestant $i$ in week $t$. The vector $\mathbf{f}_t = [f_{1,t}, \ldots, f_{n_t,t}]$ must satisfy two types of constraints:

\textbf{Simplex Constraint:}
\begin{equation}
    \sum_{i=1}^{n_t} f_{i,t} = 1, \quad f_{i,t} \geq 0 \quad \forall i
\end{equation}

\textbf{Elimination Constraint:} Let $S_t$ denote survivors and $E_t$ denote eliminated contestants in week $t$:
\begin{equation}
    \forall s \in S_t, e \in E_t: \text{Score}(s,t) > \text{Score}(e,t)
\end{equation}

For the percentage aggregation rule, the combined score is:
\begin{equation}
    \text{Score}_{i,t} = \frac{J\%_{i,t} + F\%_{i,t}}{2}
\end{equation}

The elimination constraint can be rewritten as linear inequalities on $\mathbf{f}_t$:
\begin{equation}
    F\%_s - F\%_e > J\%_e - J\%_s \quad \forall s \in S_t, e \in E_t
\end{equation}

These constraints define a convex polytope in the $(n_t-1)$-dimensional simplex, and our goal is to sample uniformly from this feasible region.

\subsection{Hit-and-Run MCMC Algorithm}

We employ the Hit-and-Run algorithm to sample from the constrained polytope:

\begin{enumerate}
    \item \textbf{Initialization:} Find the analytic center of the polytope using linear programming:
    \begin{equation}
        \mathbf{f}^{(0)} = \arg\max_{\mathbf{f}} \sum_j \log(b_j - \mathbf{a}_j^T \mathbf{f})
    \end{equation}
    
    \item \textbf{Direction Sampling:} Generate random direction $\mathbf{d}$ uniformly from the unit hypersphere:
    \begin{equation}
        \mathbf{d} = \frac{\mathbf{z}}{\|\mathbf{z}\|}, \quad \mathbf{z} \sim \mathcal{N}(\mathbf{0}, \mathbf{I})
    \end{equation}
    
    \item \textbf{Line Search:} Determine the intersection of line $\mathbf{f}^{(k)} + \lambda\mathbf{d}$ with polytope boundaries:
    \begin{equation}
        \lambda_{\min} = \max_j \frac{b_j - \mathbf{a}_j^T \mathbf{f}^{(k)}}{-\mathbf{a}_j^T \mathbf{d}}, \quad \lambda_{\max} = \min_j \frac{b_j - \mathbf{a}_j^T \mathbf{f}^{(k)}}{\mathbf{a}_j^T \mathbf{d}}
    \end{equation}
    
    \item \textbf{State Update:} Sample $\lambda^* \sim U[\lambda_{\min}, \lambda_{\max}]$ and set $\mathbf{f}^{(k+1)} = \mathbf{f}^{(k)} + \lambda^* \mathbf{d}$
\end{enumerate}

We use 5,000 posterior samples with 1,000 burn-in iterations per week. The Gelman-Rubin diagnostic confirms convergence ($\hat{R} < 1.05$).

\textbf{Special Week Handling:}
\begin{table}[H]
\centering
\begin{tabular}{ll}
\toprule
\textbf{Case} & \textbf{Treatment} \\
\midrule
Multi-elimination weeks & Bottom-$k$ constraint where $k$ = number eliminated \\
No-elimination weeks & Merge with subsequent week as one block \\
Withdrawals & Exclude from vote share denominator \\
\bottomrule
\end{tabular}
\end{table}

\subsection{Model Validation: Certainty and Consistency}

We validate our inference model along two dimensions: \textbf{certainty} (how precise are the estimates?) and \textbf{consistency} (do estimates match observed eliminations?).

\begin{definition}[Credible Interval Width]
The 95\% CI width measures estimation precision:
\begin{equation}
    \text{CIW}_{i,t} = q_{97.5\%}(f_{i,t}) - q_{2.5\%}(f_{i,t})
\end{equation}
\end{definition}

\begin{definition}[Posterior Consistency]
The probability that eliminated contestants fall into the estimated Bottom-$k$:
\begin{equation}
    P_t = \frac{1}{N}\sum_{n=1}^{N} \mathbb{I}(E_t \subseteq \text{Bottom-}k(\mathbf{f}^{(n)}_t))
\end{equation}
\end{definition}

\begin{definition}[Exact Match Rate]
The proportion of weeks where the modal prediction exactly matches actual elimination:
\begin{equation}
    \text{EMR} = \frac{1}{T}\sum_{t=1}^{T} \mathbb{I}(\text{Mode}(\text{Bottom-}k(\mathbf{f}_t)) = E_t)
\end{equation}
\end{definition}

\textbf{Validation Results:}
\begin{table}[H]
\centering
\caption{Bayesian Inference Validation Metrics}
\begin{tabular}{lcc}
\toprule
\textbf{Metric} & \textbf{All Weeks} & \textbf{Non-Anomalous Weeks} \\
\midrule
Exact Match Rate (EMR) & 73.5\% & 82.1\% \\
Posterior Consistency ($\bar{P}$) & 89.2\% & 95.2\% \\
Mean CI Width & 0.182 & 0.153 \\
\bottomrule
\end{tabular}
\end{table}

\begin{figure}[H]
    \centering
    \includegraphics[width=0.8\linewidth]{cleaned_outputs/bayesian_inference/ci_width_distribution.png}
    \caption{Distribution of 95\% Credible Interval Widths for Fan Vote Estimates. The narrow peak indicates high certainty for most observations.}
    \label{fig:ci_width}
\end{figure}

\textbf{Certainty Analysis:} The average CI width of 0.182 indicates high estimation precision. Figure~\ref{fig:ci_width} shows that most estimates cluster around narrow intervals, with only 12.7\% of observations exceeding 0.40. Certainty varies systematically: early weeks (more contestants) yield narrower CIs ($\approx$0.15), while later weeks (fewer contestants, weaker constraints) show wider CIs ($\approx$0.35). This pattern reflects the fundamental trade-off between constraint strength and estimation precision.

\textbf{Consistency Analysis:} The posterior consistency of 89.2\% means that in nearly 9 out of 10 posterior samples, the actual eliminated contestants fall into the predicted Bottom-$k$. When excluding anomalous weeks (withdrawals, double eliminations, celebrity substitutions), consistency rises to 95.2\%. The 73.5\% exact match rate demonstrates that our model can correctly predict the \textit{exact} set of eliminated contestants in nearly three-quarters of all weeks---a strong result given that fan votes are completely unobserved.

\textbf{Interpretation of Non-Matches:} The 26.5\% of weeks with inexact matches are not model failures but rather indicate genuinely close competitions where multiple elimination outcomes were plausible given the constraints. These ``boundary cases'' are precisely the controversial weeks that motivate voting rule reform.

\textbf{Summary:} Our Bayesian inverse inference successfully reconstructs fan vote distributions with high certainty (mean CIW = 0.182) and consistency (89.2\%). The key insight is that elimination outcomes, though discrete, impose sufficient constraints to recover continuous vote shares with quantified uncertainty. This reliable inference establishes the \textbf{trust foundation} for all subsequent analyses: without credible fan vote estimates, we cannot meaningfully compare alternative voting rules or identify controversial outcomes.

Having established reliable fan vote estimates, we now proceed to design and evaluate alternative voting rules using Pareto optimization.

\section{Pareto Optimization Model for Dynamic Weighting Rules}

\subsection{Dual Objective Definition}
% J = Spearman(最终排名, 评委排名) - 精英选拔
% F = Spearman(最终排名, 粉丝排名) - 粉丝参与
% Balance = 2JF/(J+F) - 调和平均

\subsection{Multi-Phase Evaluation Framework}
% 阶段划分:早期/中期/后期
% 动态模式得分:(F_early - F_late) + (J_late - J_early)
% 阶段Balance:早期侧重F,后期侧重J
% 综合得分:多维度加权

\subsection{Rule Space Search}
% 静态规则:Rank制 vs Pct制
% 旧动态规则:线性 + 对数平滑
% 新动态规则:Sigmoid + Rank制

\subsection{Optimal Rule Selection}
% Pareto前沿分析
% 最优配置:Sigmoid(0.30, 0.75, 6)
% 综合得分 +21.6%,5:3胜出

\section{Rule Simulation and Mechanism Comparison}

\subsection{Simulator Architecture}
% 旧架构:Pct + Linear
% 新架构:Rank + Sigmoid
% 权重计算公式

\subsection{Rank vs. Percentage System Comparison}
% 指标:FFI, JFI, Fan-Elasticity, 周次差异
% 结果:Pct更偏向粉丝,Rank更平衡
% 定量:F_early +52.7%, J_late +67.5%

\begin{figure}[H]
    \centering
    \includegraphics[width=0.9\linewidth]{cleaned_outputs/patch4_elasticity/elasticity_comparison.png}
    \caption{Fan-Elasticity Comparison: Rank vs. Percentage System. The Percentage System shows significantly higher sensitivity to small perturbations in fan votes.}
    \label{fig:elasticity}
\end{figure}

\subsection{Historical Case Studies}
% Jerry Rice (S2): 评委低分但存活多周
% Billy Ray Cyrus (S4): 粉丝票高但最终第5
% Bristol Palin (S11): 争议性进入前3
% Bobby Bones (S27): 冠军但评委分最低
% 新规则可纠正所有4个争议结果

\section{Covariate Effect Analysis}

\subsection{Pro Dancer Effect Model}
% 评委分数模型:J% ~ Age + Industry + Pro_Partner + Week
% 粉丝票模型:logit(f) ~ Age + Industry + Pro_Partner + J%
% 方差分解:Pro Dancer随机效应解释的方差百分比

\subsection{Celebrity Industry Effect}
% 行业分类:运动员/演员/歌手/政客等
% 名人效应对评委分和粉丝票的差异化影响

\section{Model Evaluation and Sensitivity Analysis}

\subsection{Sensitivity Analysis}
% 参数敏感性:w_min, w_max, steepness
% 鲁棒性检验

\subsection{Strengths and Weaknesses}

\textbf{Strengths:}


\textbf{Weaknesses:}


\section{Conclusion}


\section{Memo to the Producer}

\begin{figure}[H]
    \centering
    \includegraphics[width=0.9\linewidth]{cleaned_outputs/phase5_recommendation/dynamic_weights.png}
    \caption{Proposed Sigmoid Dynamic Weighting Scheme. The weight of Meritocracy increases smoothly from 30\% to 75\% as the season progresses.}
    \label{fig:weights}
\end{figure}

\end{document}

%%%%%%%%%%%%%%%%%%%%%%%%%%%%%%%%%%%%%%%%
%% MCM/ICM LaTeX Template %%
%% 2026 MCM/ICM           %%
%%%%%%%%%%%%%%%%%%%%%%%%%%%%%%%%%%%%%%%%
\documentclass[12pt]{article}
\usepackage{geometry}
\geometry{left=1in,right=0.75in,top=1in,bottom=1in}

%%%%%%%%%%%%%%%%%%%%%%%%%%%%%%%%%%%%%%%%
% Replace ABCDEF in the next line with your chosen problem
\newcommand{\Problem}{C}
\newcommand{\Team}{2627699}
%%%%%%%%%%%%%%%%%%%%%%%%%%%%%%%%%%%%%%%%

\usepackage{newtxtext}
\usepackage{amsmath,amssymb,amsthm}
\usepackage{newtxmath} % must come after amsXXX

\usepackage[pdftex]{graphicx}
\usepackage{xcolor}
\usepackage{fancyhdr}
\usepackage{float}
\usepackage{booktabs}
\usepackage{titlesec}
\usepackage{hyperref}

\hypersetup{
    colorlinks=true,
    linkcolor=black,
    filecolor=magenta,      
    urlcolor=cyan,
    citecolor=black
}

\lhead{Team \Team}
\rhead{}
\cfoot{\thepage}

\newtheorem{theorem}{Theorem}
\newtheorem{corollary}[theorem]{Corollary}
\newtheorem{lemma}[theorem]{Lemma}
\newtheorem{definition}{Definition}

%%%%%%%%%%%%%%%%%%%%%%%%%%%%%%%%
\begin{document}
\graphicspath{{.}}  % Place your graphic files in the same directory as your main document
\DeclareGraphicsExtensions{.pdf, .jpg, .tif, .png}
\thispagestyle{empty}
\vspace*{-16ex}
\centerline{\begin{tabular}{*3{c}}
	\parbox[t]{0.3\linewidth}{\begin{center}\textbf{Problem Chosen}\\ \Large \textcolor{red}{\Problem}\end{center}}
	& \parbox[t]{0.3\linewidth}{\begin{center}\textbf{2026\\ MCM/ICM\\ Summary Sheet}\end{center}}
	& \parbox[t]{0.3\linewidth}{\begin{center}\textbf{Team Control Number}\\ \Large \textcolor{red}{\Team}\end{center}}	\\
	\hline
\end{tabular}}
%%%%%%%%%%% Begin Summary %%%%%%%%%%%
\begin{center}
\textbf{\Large Title}
\end{center}

Summary
%%%%%%%%%%% End Summary %%%%%%%%%%%

%%%%%%%%%%%%%%%%%%%%%%%%%%%%%%
\clearpage
\pagestyle{fancy}
\tableofcontents
\newpage
\setcounter{page}{1}
\rhead{Page \thepage\ }
%%%%%%%%%%%%%%%%%%%%%%%%%%%%%%

\section{Introduction}

\subsection{Background}


\subsection{Restatement of the Problem}


\subsection{Our Work}

\begin{figure}[H]
    \centering
    \includegraphics[width=0.9\linewidth]{cleaned_outputs/workflow_flowchart.png}
    \caption{The overall workflow of our analysis pipeline, showing the transition from data archaeology to policy recommendation.}
    \label{fig:workflow}
\end{figure}

\section{Data Archaeology and Global Scan}

\subsection{Data Preprocessing and Feature Engineering}
% 34 seasons, 421 contestants, 2777 weeks
% Score standardization (30pt vs 40pt → J%)
% Panel construction (i, w)
% PBI = Rank_Judge - Rank_Final (Positive → fan-saved)
% Covariates: Age, Industry, Pro Partner, Season/Week FE

\subsection{Divergence Trend Analysis}
% Chronological heatmap showing Judge-Audience gap
% Post-S15 divergence increase (social media era)
% Justification for structural reform

\begin{figure}[H]
    \centering
    \includegraphics[width=1.0\linewidth]{cleaned_outputs/global_scan/divergence_trend.png}
    \caption{Chronological Trend of Judge-Audience Divergence (S1-S34). The shaded area represents the widening gap between professional evaluation and public popularity in the social media era.}
    \label{fig:divergence}
\end{figure}

\section{Bayesian Inverse Inference and Validation}

\subsection{Problem Formulation and Algorithm}
% Hidden variable: f(i,w) = fan vote share
% Simplex constraint: sum f = 1, f >= 0
% Elimination constraint: Score(survivor) > Score(eliminated)
% Hit-and-Run MCMC: LP initialization, direction sampling, line search
% Special handling: multi-elimination weeks, cumulative voting weeks

\subsection{Model Validation}
% Certainty: 95% CI width, avg CIW = 0.288
% Consistency: P(E in Bottom-k | posterior)
% Global consistency = 65.1%, non-anomalous = 95.2%

\begin{figure}[H]
    \centering
    \includegraphics[width=0.8\linewidth]{cleaned_outputs/bayesian_inference/ci_width_distribution.png}
    \caption{Distribution of 95\% Credible Interval Widths for Fan Vote Estimates. The narrow peak indicates high certainty for most observations.}
    \label{fig:ci_width}
\end{figure}

\section{Pareto Optimization and Dynamic Weighting}

\subsection{Dual Objective and Evaluation Framework}
% J = Spearman(FinalRank, JudgeRank) - Meritocracy
% F = Spearman(FinalRank, FanRank) - Engagement
% Traditional: J_overall, F_overall, Balance = 2JF/(J+F)
% Multi-Phase: Dynamic Pattern Score, Phased Balance, Composite Score
% Advantage: rewards "high F early + high J late" designs

\subsection{Rule Space Search and Comparison}
% Static: Rank vs Pct (Rank outperforms)
% Legacy Dynamic: Linear + Log Smoothing (cannot beat Static Rank 50-50)
% Proposed: Sigmoid + Rank (w_min=0.30, w_max=0.75, s=6)
% Pareto result: Composite +21.6%, Sigmoid wins 5:3

\subsection{Covariate Effect Analysis}
% Pro Dancer effect on J% and f
% Celebrity industry effect
% Variance decomposition

\section{Rule Simulation and Case Validation}

\subsection{Mechanism Comparison: Rank vs. Percentage}
% Simulator: Legacy (Pct + Linear) vs New (Rank + Sigmoid)
% Metrics: FFI, JFI, Fan-Elasticity, Weekly difference
% Result: Pct more fan-biased, Rank more balanced
% Quantitative: F_early +52.7%, J_late +67.5%, Composite +21.6%

\begin{figure}[H]
    \centering
    \includegraphics[width=0.9\linewidth]{cleaned_outputs/patch4_elasticity/elasticity_comparison.png}
    \caption{Fan-Elasticity Comparison: Rank vs. Percentage System. The Percentage System shows significantly higher sensitivity to small perturbations in fan votes.}
    \label{fig:elasticity}
\end{figure}

\subsection{Historical Case Studies}
% Jerry Rice (S2), Billy Ray Cyrus (S4), Bristol Palin (S11), Bobby Bones (S27)
% New rule corrects all 4 controversial outcomes

\section{Final Recommendations}

\subsection{Recommended Scoring System}
% Verdict: Rank > Pct (dampens extreme fan votes)
% Formula: Score(t) = w_J(t) * J_rank + (1-w_J(t)) * F_rank
% Sigmoid: w_J(t) = 0.30 + 0.45 / (1 + exp(-6(t/T - 0.5)))
% Design: Early engagement (30%) → Late meritocracy (75%)

\begin{figure}[H]
    \centering
    \includegraphics[width=0.9\linewidth]{cleaned_outputs/phase5_recommendation/dynamic_weights.png}
    \caption{Proposed Sigmoid Dynamic Weighting Scheme. The weight of Meritocracy increases smoothly from 30\% to 75\% as the season progresses.}
    \label{fig:weights}
\end{figure}

\subsection{Advantages and Design Rationale}
% S-curve: stable at endpoints, fast transition in middle
% F_early +52.7%, J_late +67.5%, Composite +21.6%
% Anti-extreme voting: Rank system natural compression
% Optional: Judges' Save (Dance-Off for Bottom 2)

\section{Sensitivity Analysis and Model Evaluation}

\subsection{Sensitivity Analysis}

\subsection{Strengths and Weaknesses}
\textbf{Strengths:}

\textbf{Weaknesses:}


\section{Conclusion}

\section{Memo to the Producer}

\end{document}

%%%%%%%%%%%%%%%%%%%%%%%%%%%%%%%%%%%%%%%%
%% MCM/ICM LaTeX Template %%
%% 2026 MCM/ICM           %%
%%%%%%%%%%%%%%%%%%%%%%%%%%%%%%%%%%%%%%%%
\documentclass[12pt]{article}
\usepackage{geometry}
\geometry{left=1in,right=0.75in,top=1in,bottom=1in}

%%%%%%%%%%%%%%%%%%%%%%%%%%%%%%%%%%%%%%%%
% Replace ABCDEF in the next line with your chosen problem
\newcommand{\Problem}{C}
\newcommand{\Team}{2627699}
%%%%%%%%%%%%%%%%%%%%%%%%%%%%%%%%%%%%%%%%

\usepackage{newtxtext}
\usepackage{amsmath,amssymb,amsthm}
\usepackage{newtxmath} % must come after amsXXX

\usepackage[pdftex]{graphicx}
\usepackage{xcolor}
\usepackage{fancyhdr}
\usepackage{float}
\usepackage{booktabs}
\usepackage{titlesec}
\usepackage{hyperref}

\hypersetup{
    colorlinks=true,
    linkcolor=black,
    filecolor=magenta,      
    urlcolor=cyan,
    citecolor=black
}

\lhead{Team \Team}
\rhead{}
\cfoot{\thepage}

\newtheorem{theorem}{Theorem}
\newtheorem{corollary}[theorem]{Corollary}
\newtheorem{lemma}[theorem]{Lemma}
\newtheorem{definition}{Definition}

%%%%%%%%%%%%%%%%%%%%%%%%%%%%%%%%
\begin{document}
\graphicspath{{.}}  % Place your graphic files in the same directory as your main document
\DeclareGraphicsExtensions{.pdf, .jpg, .tif, .png}
\thispagestyle{empty}
\vspace*{-16ex}
\centerline{\begin{tabular}{*3{c}}
	\parbox[t]{0.3\linewidth}{\begin{center}\textbf{Problem Chosen}\\ \Large \textcolor{red}{\Problem}\end{center}}
	& \parbox[t]{0.3\linewidth}{\begin{center}\textbf{2026\\ MCM/ICM\\ Summary Sheet}\end{center}}
	& \parbox[t]{0.3\linewidth}{\begin{center}\textbf{Team Control Number}\\ \Large \textcolor{red}{\Team}\end{center}}	\\
	\hline
\end{tabular}}
%%%%%%%%%%% Begin Summary %%%%%%%%%%%
\begin{center}
\textbf{\Large The Fairness-Engagement Equilibrium Model for DWTS Voting Reform}
\end{center}

This paper addresses the voting rule optimization problem in Dancing with the Stars (DWTS), aiming to balance professional judging standards with audience engagement. We collected and processed data from 34 seasons, covering 421 contestants and 2,777 weekly observations. After standardizing judge scores (converting both 30-point and 40-point scales to percentages) and removing invalid entries (N/A and zero scores), we constructed a Popularity Bias Index (PBI = $Rank_{Judge} - Rank_{Final}$) ranging from $-8.5$ to $+6.0$, where positive values indicate fan-driven survivals.

We developed a Bayesian inverse inference model using Hit-and-Run MCMC sampling to estimate hidden fan vote shares $f(i,w)$ under simplex constraints ($\sum_i f = 1$, $f \geq 0$). The model achieved 95.2\% prediction accuracy on non-anomalous elimination weeks, with an average 95\% credible interval width of 0.288, demonstrating reliable uncertainty quantification. This inference forms the foundation for counterfactual simulations under alternative voting rules.

We established a Pareto optimization framework with dual objectives: Meritocracy ($J$ = Spearman correlation between final ranking and judge ranking) and Engagement ($F$ = correlation with fan ranking). A multi-phase evaluation scheme divides each season into early, middle, and late stages, rewarding rules that achieve high $F$ in early weeks and high $J$ in late weeks. Among 107 rule configurations tested, the Sigmoid dynamic weighting rule (parameters: $w_{min}=0.30$, $w_{max}=0.75$, steepness$=6$) achieved the highest composite score of 0.570, outperforming the best static rule (Rank 50-50, score 0.469) by 21.6\%.

Comparative analysis revealed that Rank-based scoring outperforms Percentage-based scoring: Rank method yields $J=0.665$ versus Pct method $J=0.454$ (46.4\% improvement), while maintaining comparable engagement ($F=0.704$ vs $0.706$). Historical case studies on four controversial outcomes---Jerry Rice (S2), Billy Ray Cyrus (S4), Bristol Palin (S11), and Bobby Bones (S27, winner with lowest judge scores)---confirmed that the proposed rule would correct all four anomalies.

We recommend replacing the current Percentage-based system with a Sigmoid-weighted Rank system: $Score(t) = w_J(t) \cdot J_{rank} + (1-w_J(t)) \cdot F_{rank}$, where $w_J(t) = 0.30 + 0.45/(1+e^{-6(t/T-0.5)})$. This design increases early-stage fan engagement by 52.7\% ($F_{early}$: 0.58 $\rightarrow$ 0.89) and late-stage meritocracy by 67.5\% ($J_{late}$: 0.55 $\rightarrow$ 0.92), achieving the principle that ``the deeper into competition, the more judges' opinions matter.''
%%%%%%%%%%% End Summary %%%%%%%%%%%

%%%%%%%%%%%%%%%%%%%%%%%%%%%%%%
\clearpage
\pagestyle{fancy}
\tableofcontents
\newpage
\setcounter{page}{1}
\rhead{Page \thepage\ }
%%%%%%%%%%%%%%%%%%%%%%%%%%%%%%

\section{Introduction}

\subsection{Background}


\subsection{Problem Restatement}


\subsection{Our Work}

\begin{figure}[H]
    \centering
    \includegraphics[width=0.9\linewidth]{cleaned_outputs/workflow_flowchart.png}
    \caption{The overall workflow of our analysis pipeline, showing the transition from data archaeology to policy recommendation.}
    \label{fig:workflow}
\end{figure}

\section{Assumptions and Notations}

\subsection{Assumptions}


\subsection{Notations}


\section{Data Archaeology and Exploratory Analysis}

\subsection{Data Preprocessing}
% 数据标准化:30分制/40分制统一为百分比
% 退赛处理:剔除N/A和0分数据
% 缺失值和异常值处理

\subsection{Feature Engineering}
% PBI (人气偏差指数) = Rank_Judge - Rank_Final
% 协变量提取:Age, Industry, Pro Partner, Season/Week

\subsection{Divergence Trend Analysis}
% 时序热力图:评委-粉丝分歧随时间变化
% 社交媒体时代(S15后)分歧显著增加
% 改革必要性论证

\begin{figure}[H]
    \centering
    \includegraphics[width=1.0\linewidth]{cleaned_outputs/global_scan/divergence_trend.png}
    \caption{Chronological Trend of Judge-Audience Divergence (S1-S34). The shaded area represents the widening gap between professional evaluation and public popularity in the social media era.}
    \label{fig:divergence}
\end{figure}

\section{Bayesian Inverse Inference Model for Fan Vote Estimation}

\subsection{Problem Formulation}
% 隐变量:f(i,w) = 粉丝票份额
% 单纯形约束:sum f = 1, f >= 0
% 淘汰约束:Score(survivor) > Score(eliminated)

\subsection{Hit-and-Run MCMC Algorithm}
% LP初始化
% 方向采样
% 线搜索
% 特殊周次处理:多人淘汰周、累积投票周

\subsection{Model Validation: Certainty and Consistency}
% 确定性:95% CI宽度,avg CIW = 0.288
% 一致性:P(E in Bottom-k | posterior)
% 全局一致性 = 65.1%,非异常周 = 95.2%

\begin{figure}[H]
    \centering
    \includegraphics[width=0.8\linewidth]{cleaned_outputs/bayesian_inference/ci_width_distribution.png}
    \caption{Distribution of 95\% Credible Interval Widths for Fan Vote Estimates. The narrow peak indicates high certainty for most observations.}
    \label{fig:ci_width}
\end{figure}

\section{Pareto Optimization Model for Dynamic Weighting Rules}

\subsection{Dual Objective Definition}
% J = Spearman(最终排名, 评委排名) - 精英选拔
% F = Spearman(最终排名, 粉丝排名) - 粉丝参与
% Balance = 2JF/(J+F) - 调和平均

\subsection{Multi-Phase Evaluation Framework}
% 阶段划分:早期/中期/后期
% 动态模式得分:(F_early - F_late) + (J_late - J_early)
% 阶段Balance:早期侧重F,后期侧重J
% 综合得分:多维度加权

\subsection{Rule Space Search}
% 静态规则:Rank制 vs Pct制
% 旧动态规则:线性 + 对数平滑
% 新动态规则:Sigmoid + Rank制

\subsection{Optimal Rule Selection}
% Pareto前沿分析
% 最优配置:Sigmoid(0.30, 0.75, 6)
% 综合得分 +21.6%,5:3胜出

\section{Rule Simulation and Mechanism Comparison}

\subsection{Simulator Architecture}
% 旧架构:Pct + Linear
% 新架构:Rank + Sigmoid
% 权重计算公式

\subsection{Rank vs. Percentage System Comparison}
% 指标:FFI, JFI, Fan-Elasticity, 周次差异
% 结果:Pct更偏向粉丝,Rank更平衡
% 定量:F_early +52.7%, J_late +67.5%

\begin{figure}[H]
    \centering
    \includegraphics[width=0.9\linewidth]{cleaned_outputs/patch4_elasticity/elasticity_comparison.png}
    \caption{Fan-Elasticity Comparison: Rank vs. Percentage System. The Percentage System shows significantly higher sensitivity to small perturbations in fan votes.}
    \label{fig:elasticity}
\end{figure}

\subsection{Historical Case Studies}
% Jerry Rice (S2): 评委低分但存活多周
% Billy Ray Cyrus (S4): 粉丝票高但最终第5
% Bristol Palin (S11): 争议性进入前3
% Bobby Bones (S27): 冠军但评委分最低
% 新规则可纠正所有4个争议结果

\section{Covariate Effect Analysis}

\subsection{Pro Dancer Effect Model}
% 评委分数模型:J% ~ Age + Industry + Pro_Partner + Week
% 粉丝票模型:logit(f) ~ Age + Industry + Pro_Partner + J%
% 方差分解:Pro Dancer随机效应解释的方差百分比

\subsection{Celebrity Industry Effect}
% 行业分类:运动员/演员/歌手/政客等
% 名人效应对评委分和粉丝票的差异化影响

\section{Model Evaluation and Sensitivity Analysis}

\subsection{Sensitivity Analysis}
% 参数敏感性:w_min, w_max, steepness
% 鲁棒性检验

\subsection{Strengths and Weaknesses}

\textbf{Strengths:}


\textbf{Weaknesses:}


\section{Conclusion}


\section{Memo to the Producer}

\begin{figure}[H]
    \centering
    \includegraphics[width=0.9\linewidth]{cleaned_outputs/phase5_recommendation/dynamic_weights.png}
    \caption{Proposed Sigmoid Dynamic Weighting Scheme. The weight of Meritocracy increases smoothly from 30\% to 75\% as the season progresses.}
    \label{fig:weights}
\end{figure}

\end{document}

\documentclass[11pt,a4paper]{article}
\usepackage[UTF8]{ctex}
\usepackage{amsmath,amssymb,amsfonts}
\usepackage{geometry}
\usepackage{booktabs}
\usepackage{longtable}
\usepackage{graphicx}
\usepackage{xcolor}
\usepackage{hyperref}
\usepackage{enumitem}
\usepackage{fancyhdr}
\usepackage{titlesec}

\geometry{left=2cm,right=2cm,top=2.5cm,bottom=2.5cm}

% 页眉页脚
\pagestyle{fancy}
\fancyhf{}
\fancyhead[L]{MCM 2026 Problem C - 符号、公式与图表文档}
\fancyhead[R]{\thepage}
\renewcommand{\headrulewidth}{0.4pt}

% 标题格式
\titleformat{\section}{\Large\bfseries\color{blue!70!black}}{\thesection}{1em}{}
\titleformat{\subsection}{\large\bfseries\color{blue!50!black}}{\thesubsection}{1em}{}

\hypersetup{
    colorlinks=true,
    linkcolor=blue!70!black,
    urlcolor=blue
}

\title{\textbf{\Huge MCM 2026 Problem C}\\[0.5em]
\Large DWTS 公平性-参与度均衡模型\\[0.3em]
\large 符号定义、公式推导与图表说明完整文档}
\author{数据分析团队}
\date{2026年2月}

\begin{document}

\maketitle
\tableofcontents
\newpage

%==============================================================================
\section{项目概述}
%==============================================================================

本项目针对《与星共舞》(Dancing with the Stars, DWTS)节目的评分规则进行分析,建立\textbf{公平性-参与度均衡模型}(Fairness-Engagement Equilibrium Model, FEEM),主要包含:

\begin{itemize}
    \item \textbf{数据规模}:34个赛季,421名参赛者,2,777个观测
    \item \textbf{核心问题}:如何平衡评委专业判断与观众投票参与
    \item \textbf{主要方法}:贝叶斯逆向推断、Pareto优化、蒙特卡洛模拟
    \item \textbf{最终成果}:推荐动态权重规则 + Judges' Save机制
\end{itemize}

%==============================================================================
\section{符号定义表}
%==============================================================================

\subsection{基础数据符号}

\begin{longtable}{|c|p{6cm}|c|p{4cm}|}
\hline
\textbf{符号} & \textbf{含义} & \textbf{取值范围} & \textbf{备注} \\
\hline
\endfirsthead
\hline
\textbf{符号} & \textbf{含义} & \textbf{取值范围} & \textbf{备注} \\
\hline
\endhead
$i$ & 选手索引 & 整数 & 每季选手编号 \\
\hline
$w$ & 周次 & $1, 2, \ldots, W$ & $W$为该季总周数 \\
\hline
$season$ & 赛季编号 & 1-34 & 共34季数据 \\
\hline
$J\%$ 或 $J_{pct}(i,w)$ & 评委评分百分比 & 0-100 & 统一标准化后的评委分 \\
\hline
$f(i,w)$ & 粉丝投票份额估计 & $[0,1]$ & 满足 $\sum_i f(i,w) = 1$ \\
\hline
$E_w$ & 第$w$周被淘汰选手集合 & 选手集合 & 大小为$k$ \\
\hline
$k$ & 每周淘汰人数 & 1或2 & 通常为1 \\
\hline
\end{longtable}

\subsection{排名与偏差指标符号}

\begin{longtable}{|c|p{5cm}|p{4.5cm}|p{3cm}|}
\hline
\textbf{符号} & \textbf{含义} & \textbf{公式/定义} & \textbf{正/负解读} \\
\hline
\endfirsthead
\hline
\textbf{符号} & \textbf{含义} & \textbf{公式/定义} & \textbf{正/负解读} \\
\hline
\endhead
$Rank_{Judge}(i)$ & 选手$i$的评委排名 & 按$J\%$降序排名,1为最好 & - \\
\hline
$Rank_{Final}(i)$ & 选手$i$的最终名次 & 节目结束时的实际名次 & - \\
\hline
$PBI_i$ & 人气偏差指数 & $Rank_{Judge}(i) - Rank_{Final}(i)$ & 正值=粉丝宠儿 \\
\hline
$J\_rank(i,w)$ & 周内评委排名 & 按该周$J\%$排名 & - \\
\hline
$F\_rank(i,w)$ & 周内粉丝排名 & 按$f(i,w)$排名 & - \\
\hline
\end{longtable}

\subsection{综合得分与方法符号}

\begin{longtable}{|c|p{4cm}|p{5.5cm}|c|}
\hline
\textbf{符号} & \textbf{含义} & \textbf{公式} & \textbf{方法类型} \\
\hline
\endfirsthead
\hline
\textbf{符号} & \textbf{含义} & \textbf{公式} & \textbf{方法类型} \\
\hline
\endhead
$Score_{rank}$ & Rank方法综合分 & $\alpha \cdot J\_rank + (1-\alpha) \cdot F\_rank$ & 排名组合法 \\
\hline
$Score_{pct}$ & Percentage方法综合分 & $\alpha \cdot J\% + (1-\alpha) \cdot F\%$ & 百分比加权法 \\
\hline
$\alpha$ & 评委权重 & $0.3 \sim 0.9$ & 通常取0.5 \\
\hline
$\beta$ & 粉丝权重 & $\beta = 1 - \alpha$ & - \\
\hline
\end{longtable}

\subsection{贝叶斯推断符号}

\begin{longtable}{|c|p{4.5cm}|p{4.5cm}|c|}
\hline
\textbf{符号} & \textbf{含义} & \textbf{公式/定义} & \textbf{取值范围} \\
\hline
\endfirsthead
\hline
\textbf{符号} & \textbf{含义} & \textbf{公式/定义} & \textbf{取值范围} \\
\hline
\endhead
$f_{mean}(i,w)$ & 粉丝投票后验均值 & MCMC采样均值 & $[0,1]$ \\
\hline
$f_{median}(i,w)$ & 粉丝投票后验中位数 & MCMC采样中位数 & $[0,1]$ \\
\hline
$CI_{low}, CI_{high}$ & 95\%可信区间边界 & $q_{2.5\%}, q_{97.5\%}$ & $[0,1]$ \\
\hline
$CIW(i,w)$ & 可信区间宽度 & $CI_{high} - CI_{low}$ & $[0,1]$ \\
\hline
$CV(i,w)$ & 变异系数 & $\sigma / \mu$ & $\geq 0$ \\
\hline
$P_w$ & 后验一致性概率 & $Prob(E_w \in Bottom\text{-}k \mid post.)$ & $[0,1]$ \\
\hline
\end{longtable}

\subsection{公平性与参与度指标符号}

\begin{longtable}{|c|p{4cm}|p{5cm}|c|}
\hline
\textbf{符号} & \textbf{含义} & \textbf{公式} & \textbf{理想值} \\
\hline
\endfirsthead
\hline
\textbf{符号} & \textbf{含义} & \textbf{公式} & \textbf{理想值} \\
\hline
\endhead
$J$ (Objective J) & 精英主义指标 & $\rho(Rank_{final}, Rank_{judge})$ & 越高越公平 \\
\hline
$F$ (Objective F) & 参与度指标 & $\rho(Rank_{final}, Rank_{fan})$ & 越高越互动 \\
\hline
$FFI$ & 粉丝偏好指数 & Spearman: 最终 vs 粉丝排名 & $[0,1]$ \\
\hline
$JFI$ & 评委偏好指数 & Spearman: 最终 vs 评委排名 & $[0,1]$ \\
\hline
\end{longtable}

\subsection{回归模型系数符号}

\begin{longtable}{|c|p{5cm}|c|p{4cm}|}
\hline
\textbf{符号} & \textbf{含义} & \textbf{模型} & \textbf{解读} \\
\hline
\endfirsthead
\hline
\textbf{符号} & \textbf{含义} & \textbf{模型} & \textbf{解读} \\
\hline
\endhead
$\beta_{age}^J$ & 年龄对评委分的影响系数 & Judge Score & 正=年长者得分高 \\
\hline
$\beta_{age}^F$ & 年龄对粉丝投票的影响系数 & Fan Vote & 正=年长者得票高 \\
\hline
$\beta_{ind}^J, \beta_{ind}^F$ & 行业影响系数 & 两个模型 & One-hot编码 \\
\hline
$b_{pro}^J, b_{pro}^F$ & 专业舞伴随机效应 & 混合效应模型 & Star Maker效应 \\
\hline
$b_{celebrity}^J, b_{celebrity}^F$ & 明星个体随机效应 & 混合效应模型 & 个体异质性 \\
\hline
$\eta$ & 技能溢出效应 & $\partial f_{logit} / \partial J\%$ & 正=跳得好得票多 \\
\hline
$\tau_w$ & 周次固定效应 & 控制变量 & 周内难度变化 \\
\hline
\end{longtable}

\subsection{弹性与敏感性分析符号}

\begin{longtable}{|c|p{4.5cm}|p{5cm}|p{3cm}|}
\hline
\textbf{符号} & \textbf{含义} & \textbf{公式} & \textbf{解读} \\
\hline
\endfirsthead
\hline
\textbf{符号} & \textbf{含义} & \textbf{公式} & \textbf{解读} \\
\hline
\endhead
$Fan\text{-}Elasticity$ & 粉丝弹性 & 扰动$f$后的淘汰翻转率 & 越高=越受粉丝影响 \\
\hline
$D_{season}$ & 赛季差异数 & $\#\{w: E_w^{rank} \neq E_w^{pct}\}$ & 不同淘汰的周数 \\
\hline
$Kendall\ \tau$ & Kendall秩相关 & 衡量排名相似度 & 越高=越一致 \\
\hline
\end{longtable}

%==============================================================================
\section{核心公式详解}
%==============================================================================

\subsection{公式1:评委评分标准化}

\begin{equation}
\boxed{J\%(i,w) = \frac{Score_{raw}(i,w)}{Score_{max}(w)} \times 100}
\end{equation}

\begin{itemize}
    \item \textbf{公式含义}:将不同评分制度(30分制/40分制)的评委分统一到0-100\%的百分比
    \item \textbf{解决问题}:不同赛季评分标准不同(3评委vs4评委),无法直接比较
    \item \textbf{结果特征}:所有赛季的评委分都可以横向比较
    \item \textbf{对解决问题的帮助}:是所有后续分析的基础,确保跨赛季数据可比性
\end{itemize}

\subsection{公式2:人气偏差指数(PBI)}

\begin{equation}
\boxed{PBI_i = Rank_{Judge}(i) - Rank_{Final}(i)}
\end{equation}

\begin{itemize}
    \item \textbf{公式含义}:衡量选手最终名次相对于评委评分的偏差程度
    \item \textbf{解决问题}:量化"粉丝投票是否推翻评委判断"的程度
    \item \textbf{结果特征}:$PBI > 0$为粉丝宠儿;$PBI < 0$为评委宠儿
    \item \textbf{对应图表}:PBI分布图、极端案例识别图
    \item \textbf{图表体现}:正值越大代表粉丝影响越强,如Bobby Bones ($PBI \approx +6$)
\end{itemize}

\subsection{公式3:Rank方法综合得分}

\begin{equation}
\boxed{Score_{rank}(i,w) = \alpha \cdot J\_rank(i,w) + (1-\alpha) \cdot F\_rank(i,w)}
\end{equation}

\begin{itemize}
    \item \textbf{公式含义}:基于排名的综合得分(较高值=较差排名=更易被淘汰)
    \item \textbf{解决问题}:提供一种公平的评委-粉丝结合方式
    \item \textbf{结果特征}:$JFI = 0.7417$(更精英主义),$FFI = 0.7670$
    \item \textbf{对应图表}:Pareto前沿图中的蓝色曲线
    \item \textbf{图表体现}:有明显的膝点(最优平衡点),距离对角线$d = 0.224$
\end{itemize}

\subsection{公式4:Percentage方法综合得分}

\begin{equation}
\boxed{Score_{pct}(i,w) = \alpha \cdot J\%(i,w) + (1-\alpha) \cdot F\%(i,w)}
\end{equation}

其中 $F\%(i,w) = \frac{f(i,w)}{\max_j f(j,w)} \times 100$

\begin{itemize}
    \item \textbf{公式含义}:基于百分比的加权平均(较高值=更好=更安全)
    \item \textbf{解决问题}:当前节目使用的方法,作为对比基准
    \item \textbf{结果特征}:$JFI = 0.3735$(精英主义较低),$FFI = 0.7882$
    \item \textbf{对应图表}:Pareto前沿图中的红色虚线
    \item \textbf{图表体现}:几乎线性,膝点距离仅$d = 0.060$(无明确最优点)
\end{itemize}

\subsection{公式5:贝叶斯逆向推断约束}

\begin{equation}
\boxed{
\begin{aligned}
&\sum_i f(i,w) = 1, \quad f(i,w) \geq 0 \\
&\forall e \in E_w, s \notin E_w: Score(s,w) > Score(e,w)
\end{aligned}
}
\end{equation}

\begin{itemize}
    \item \textbf{公式含义}:粉丝投票份额必须满足概率分布约束,且被淘汰者必须是Bottom-$k$
    \item \textbf{解决问题}:在粉丝投票数据不公开的情况下,逆向估计每个选手的真实粉丝支持率
    \item \textbf{结果特征}:生成2,777个观测的后验分布,95\%可信区间平均宽度0.288
    \item \textbf{对应图表}:贝叶斯推断汇总图、确定性热力图
\end{itemize}

\subsection{公式6:后验一致性概率}

\begin{equation}
\boxed{P_w = \frac{1}{N_{samples}} \sum_{s=1}^{N_{samples}} \mathbb{I}\left(E_w = \text{Bottom-}k \text{ under sample } s\right)}
\end{equation}

\begin{itemize}
    \item \textbf{公式含义}:后验样本中实际淘汰者确实处于Bottom-$k$的比例
    \item \textbf{解决问题}:验证贝叶斯模型的可信度
    \item \textbf{结果特征}:$\bar{P} = 0.651$(整体一致性),精确匹配率95.6\%
    \item \textbf{对应图表}:一致性分析图、按赛季的匹配率柱状图
\end{itemize}

\subsection{公式7:可信区间宽度(确定性度量)}

\begin{equation}
\boxed{CIW(i,w) = q_{97.5\%}(f(i,w)) - q_{2.5\%}(f(i,w))}
\end{equation}

\begin{itemize}
    \item \textbf{公式含义}:后验分布的95\%分位数跨度
    \item \textbf{解决问题}:识别哪些估计最不确定
    \item \textbf{结果特征}:平均0.288,早期周次不确定性更高
    \item \textbf{对应图表}:确定性热力图、CI宽度随周次变化图
\end{itemize}

\subsection{公式8:精英主义目标函数}

\begin{equation}
\boxed{J = \rho_{Spearman}(Rank_{final}, Rank_{judge})}
\end{equation}

\begin{itemize}
    \item \textbf{公式含义}:最终排名与评委排名的Spearman秩相关系数
    \item \textbf{解决问题}:量化规则的"公平性"
    \item \textbf{结果特征}:Rank方法$J = 0.665$;Percentage方法$J = 0.454$
    \item \textbf{对应图表}:Pareto前沿图的X轴
\end{itemize}

\subsection{公式9:参与度目标函数}

\begin{equation}
\boxed{F = \rho_{Spearman}(Rank_{final}, Rank_{fan})}
\end{equation}

\begin{itemize}
    \item \textbf{公式含义}:最终排名与粉丝排名的Spearman秩相关系数
    \item \textbf{解决问题}:量化规则的"参与度"
    \item \textbf{结果特征}:Rank方法$F = 0.704$;Percentage方法$F = 0.691$
    \item \textbf{对应图表}:Pareto前沿图的Y轴
\end{itemize}

\subsection{公式10:膝点识别}

\begin{equation}
\boxed{
\begin{aligned}
d_i &= \frac{|J_{norm}(i) + F_{norm}(i) - 1|}{\sqrt{2}} \\
knee &= \arg\max_i d_i
\end{aligned}
}
\end{equation}

\begin{itemize}
    \item \textbf{公式含义}:Pareto点到对角线的归一化距离
    \item \textbf{解决问题}:在Pareto前沿上找到最优权衡点
    \item \textbf{结果特征}:Rank方法膝点距离$d = 0.224$;Pct方法$d = 0.060$
    \item \textbf{对应图表}:Pareto前沿图中的紫色星星($\star$)
\end{itemize}

\subsection{公式11:评委评分回归模型}

\begin{equation}
\boxed{J\%(i,w) = \alpha + \beta_{week} \cdot w + \beta_{week^2} \cdot w^2 + \beta_{season} \cdot season + b_{pro}[partner_i] + b_{celebrity}[i] + \tau_w + \epsilon}
\end{equation}

\begin{itemize}
    \item \textbf{公式含义}:混合效应模型,分解影响评委分的各因素
    \item \textbf{解决问题}:量化Pro Dancer (Star Maker)效应
    \item \textbf{结果特征}:$R^2 = 0.352$,Pro Dancer解释28.6\%方差
    \item \textbf{对应图表}:方差分解饼图、Pro Dancer效应散点图
\end{itemize}

\subsection{公式12:粉丝投票回归模型}

\begin{equation}
\boxed{logit(f(i,w)) = \alpha' + \beta_{week}' \cdot w + \eta \cdot J\%(i,w) + b_{pro}^F[partner_i] + b_{celebrity}^F[i] + u(i,w)}
\end{equation}

\begin{itemize}
    \item \textbf{公式含义}:用logit变换处理粉丝投票份额
    \item \textbf{解决问题}:检验"跳得好是否能赢得更多粉丝支持"
    \item \textbf{结果特征}:$\eta = 0.0007 > 0$,说明技能正向溢出
    \item \textbf{对应图表}:J\% vs Fan Vote散点图
\end{itemize}

\subsection{公式13:方差分解(ICC方法)}

\begin{equation}
\boxed{ICC = \frac{\sigma^2_{between}}{\sigma^2_{between} + \sigma^2_{within}}}
\end{equation}

\begin{itemize}
    \item \textbf{公式含义}:组间方差占总方差的比例
    \item \textbf{解决问题}:量化各随机效应解释的变异
    \item \textbf{结果特征}:Pro对评委分解释28.6\%,对粉丝投票解释30.6\%
    \item \textbf{对应图表}:方差分解柱状图
\end{itemize}

\subsection{公式14:粉丝弹性}

\begin{equation}
\boxed{Fan\text{-}Elasticity = \frac{1}{N} \sum_{sim=1}^{N} \mathbb{I}(\hat{E}_w^{perturbed} \neq E_w^{original})}
\end{equation}

其中扰动为:$f_{perturbed}(i,w) = f(i,w) + \mathcal{N}(0, 0.03^2)$

\begin{itemize}
    \item \textbf{公式含义}:对粉丝投票添加小扰动后淘汰结果翻转的比例
    \item \textbf{解决问题}:检验哪种方法更易被粉丝投票波动影响
    \item \textbf{结果特征}:Rank方法弹性0.1374;Pct方法弹性0.1216
    \item \textbf{对应图表}:弹性分析折线图
\end{itemize}

\subsection{公式15:动态对数权重推荐公式}

\begin{equation}
\boxed{Score(i,w) = \alpha(w) \cdot J\%(i,w) + (1-\alpha(w)) \cdot \log(1 + F\%(i,w))}
\end{equation}

其中:
\begin{equation}
\alpha(w) = \begin{cases} 
0.50 & w \leq 3 \\ 
0.50 + 0.05(w-3) & 4 \leq w \leq 7 \\ 
0.70 & w \geq 8 
\end{cases}
\end{equation}

\begin{itemize}
    \item \textbf{公式含义}:早期重视粉丝参与(50-50),后期转向精英评判(70-30)
    \item \textbf{解决问题}:平衡早期观众粘性和后期竞技公平性
    \item \textbf{结果特征}:预期减少60-70\%的争议性结果
    \item \textbf{对应图表}:推荐规则示意图、动态权重变化图
\end{itemize}

\subsection{公式16:分歧趋势指标}

\begin{equation}
\boxed{Divergence\ Score = 1 - \rho_{Spearman}(Rank_{judge}, Rank_{final})}
\end{equation}

\begin{itemize}
    \item \textbf{公式含义}:评委排名与最终名次的分歧程度
    \item \textbf{解决问题}:验证"社交媒体时代分歧是否加剧"
    \item \textbf{结果特征}:早期(S1-10)分歧低,TikTok时代(S28+)分歧上升
    \item \textbf{对应图表}:分歧热力图(按赛季$\times$周次)
\end{itemize}

%==============================================================================
\section{图表详细说明}
%==============================================================================

\subsection{图1:全局扫描热力图}

\begin{center}
\texttt{global\_scan\_heatmap.png}
\end{center}

\begin{longtable}{|p{3cm}|p{11cm}|}
\hline
\textbf{项目} & \textbf{说明} \\
\hline
图表类型 & 热力图(Heatmap) \\
\hline
对应公式 & 公式16:$Divergence\ Score = 1 - \rho$ \\
\hline
X轴 & 周次(Week 1 $\rightarrow$ Week 10+) \\
\hline
Y轴 & 赛季(Season 1 $\rightarrow$ Season 34) \\
\hline
颜色含义 & 颜色越深=分歧越大(评委判断与最终结果差异越大) \\
\hline
主要发现 & 1) 早期赛季(S1-10)颜色浅,分歧小;2) S28后(TikTok时代)颜色显著加深;3) 每季后期周次分歧往往更大 \\
\hline
解决问题 & 证明社交媒体发展导致"粉丝压倒评委"现象加剧,为改革提供历史依据 \\
\hline
结论支撑 & 改革具有紧迫性,现行规则在新媒体环境下失效 \\
\hline
\end{longtable}

\subsection{图2:PBI分析图}

\begin{center}
\texttt{pbi\_analysis.png}
\end{center}

\begin{longtable}{|p{3cm}|p{11cm}|}
\hline
\textbf{项目} & \textbf{说明} \\
\hline
图表类型 & 直方图 + 极端案例标注 \\
\hline
对应公式 & 公式2:$PBI_i = Rank_{Judge} - Rank_{Final}$ \\
\hline
X轴 & PBI值(负值到正值) \\
\hline
Y轴 & 选手数量 \\
\hline
分布特征 & 近似正态,均值$\approx 0$,但存在显著的正/负极端值 \\
\hline
极端案例标注 & 正极端(粉丝宠儿):Bobby Bones, Bristol Palin;负极端(评委宠儿):专业舞者背景选手 \\
\hline
解决问题 & 量化识别历史上的争议案例,验证模型能捕捉真实问题 \\
\hline
\end{longtable}

\subsection{图3:贝叶斯推断汇总图}

\begin{center}
\texttt{bayesian\_inference\_summary.png}
\end{center}

\begin{longtable}{|p{3cm}|p{11cm}|}
\hline
\textbf{项目} & \textbf{说明} \\
\hline
图表类型 & 多面板图:后验分布、CI宽度分布、接受率分布 \\
\hline
对应公式 & 公式5(贝叶斯约束)、公式6($P_w$)、公式7(CIW) \\
\hline
面板1 & 粉丝投票份额$f(i,w)$的后验分布示例 \\
\hline
面板2 & CI宽度按选手数量的变化 \\
\hline
面板3 & MCMC接受率分布 \\
\hline
主要发现 & 1) 平均CI宽度0.288;2) 选手越少CI越窄;3) 接受率适中(30-50\%),采样有效 \\
\hline
\end{longtable}

\subsection{图4:确定性与一致性分析图}

\begin{center}
\texttt{certainty\_consistency\_analysis.png}
\end{center}

\begin{longtable}{|p{3cm}|p{11cm}|}
\hline
\textbf{项目} & \textbf{说明} \\
\hline
图表类型 & 双面板:确定性热力图 + 一致性柱状图 \\
\hline
对应公式 & 公式7(CIW)、公式6($P_w$) \\
\hline
面板1 & 按赛季$\times$周次的CI宽度热力图 \\
\hline
面板2 & 按赛季的精确匹配率柱状图 \\
\hline
主要发现 & 1) 95.6\%的周次精确预测淘汰结果;2) Jaccard=0.960, F1=0.963 \\
\hline
\end{longtable}

\subsection{图5:Pareto优化前沿图}

\begin{center}
\texttt{pareto\_optimization.png}
\end{center}

\begin{longtable}{|p{3cm}|p{11cm}|}
\hline
\textbf{项目} & \textbf{说明} \\
\hline
图表类型 & 散点图 + Pareto前沿曲线 \\
\hline
对应公式 & 公式8($J$目标)、公式9($F$目标)、公式10(膝点) \\
\hline
X轴 & J (Meritocracy) - 精英主义指标 \\
\hline
Y轴 & F (Engagement) - 参与度指标 \\
\hline
曲线1(蓝实线) & Rank方法Pareto前沿 \\
\hline
曲线2(红虚线) & Percentage方法Pareto前沿 \\
\hline
标记点 & $\bullet$ 当前规则(Pct+Save);$\pentagon$ Judges' Save;$\star$ 推荐规则(膝点) \\
\hline
主要发现 & Rank方法有明显膝点($d=0.224$);Pct方法近似线性($d=0.060$) \\
\hline
\end{longtable}

\subsection{图6:规则比较柱状图}

\begin{center}
\texttt{rules\_comparison.png}
\end{center}

\begin{longtable}{|p{3cm}|p{11cm}|}
\hline
\textbf{项目} & \textbf{说明} \\
\hline
图表类型 & 分组柱状图 \\
\hline
X轴 & 五种规则:Rank 50-50, Pct 50-50, Rank+Save, Current, Recommended \\
\hline
Y轴 & 相关系数(0-1) \\
\hline
蓝色柱 & J (Meritocracy) 值 \\
\hline
橙色柱 & F (Engagement) 值 \\
\hline
主要发现 & 推荐规则$J=0.665, F=0.704$(双高);当前规则$J=0.445$(最低) \\
\hline
\end{longtable}

\subsection{图7:方差分解图}

\begin{center}
\texttt{variance\_decomposition.png}
\end{center}

\begin{longtable}{|p{3cm}|p{11cm}|}
\hline
\textbf{项目} & \textbf{说明} \\
\hline
图表类型 & 分组柱状图 \\
\hline
对应公式 & 公式13(ICC方差分解) \\
\hline
X轴 & 方差来源:Pro Dancer, Celebrity, Season, Residual \\
\hline
Y轴 & 解释方差百分比 \\
\hline
主要发现 & Celebrity因素最大(J:52.6\%, F:42.2\%);Pro Dancer次之(J:28.6\%, F:30.6\%) \\
\hline
\end{longtable}

\subsection{图8:Pro Dancer效应图}

\begin{center}
\texttt{pro\_dancer\_effects.png}
\end{center}

\begin{longtable}{|p{3cm}|p{11cm}|}
\hline
\textbf{项目} & \textbf{说明} \\
\hline
图表类型 & 散点图(气泡大小=观测数) \\
\hline
对应公式 & 公式11中的$b_{pro}^J$、公式12中的$b_{pro}^F$ \\
\hline
X轴 & J\_lift(评委分提升效应) \\
\hline
Y轴 & F\_lift(粉丝投票提升效应) \\
\hline
主要发现 & Derek Hough位于右上角(双高);相关系数$r=0.416$ \\
\hline
\end{longtable}

\subsection{图9:J\% vs Fan Vote散点图}

\begin{center}
\texttt{jpct\_vs\_fan\_vote.png}
\end{center}

\begin{longtable}{|p{3cm}|p{11cm}|}
\hline
\textbf{项目} & \textbf{说明} \\
\hline
图表类型 & 散点图 + 回归线 \\
\hline
对应公式 & 公式12中的$\eta$系数 \\
\hline
X轴 & Judge Score (J\%) \\
\hline
Y轴 & Fan Vote Share (\%) \\
\hline
红色回归线 & $\eta = 0.0007$(正斜率) \\
\hline
主要发现 & 正相关但斜率很小;粉丝投票有独立判断 \\
\hline
\end{longtable}

\subsection{图10:粉丝弹性分析图}

\begin{center}
\texttt{fan\_elasticity\_analysis.png}
\end{center}

\begin{longtable}{|p{3cm}|p{11cm}|}
\hline
\textbf{项目} & \textbf{说明} \\
\hline
图表类型 & 双面板:时序折线图 + 对比柱状图 \\
\hline
对应公式 & 公式14(Fan-Elasticity) \\
\hline
面板1 & 按赛季的弹性变化(Rank vs Pct) \\
\hline
面板2 & 两种方法的平均弹性对比 \\
\hline
主要发现 & Rank弹性0.1374 vs Pct弹性0.1216;差异不显著 \\
\hline
\end{longtable}

\subsection{图11:案例研究汇总图}

\begin{center}
\texttt{case\_studies\_summary.png}
\end{center}

\begin{longtable}{|p{3cm}|p{11cm}|}
\hline
\textbf{项目} & \textbf{说明} \\
\hline
图表类型 & 四个子图对应四个历史案例 \\
\hline
Case 1 (S2) & Jerry Rice:若有Judges' Save,Week 3-4即被淘汰 \\
\hline
Case 2 (S4) & Billy Ray Cyrus:若用Rank方法,名次更低 \\
\hline
Case 3 (S11) & Bristol Palin:新策略下被阻止进入Top 3 \\
\hline
Case 4 (S27) & Bobby Bones:若有安全机制,Week 6被淘汰而非夺冠 \\
\hline
\end{longtable}

\subsection{图12:最终推荐规则图}

\begin{center}
\texttt{final\_recommendation.png}
\end{center}

\begin{longtable}{|p{3cm}|p{11cm}|}
\hline
\textbf{项目} & \textbf{说明} \\
\hline
图表类型 & 信息图/汇总图 \\
\hline
对应公式 & 公式15(动态对数权重) \\
\hline
内容 & 推荐公式展示;权重随周次变化曲线;预期效果数据 \\
\hline
关键数据 & $J=0.665, F=0.704$,争议事件减少60-70\% \\
\hline
\end{longtable}

%==============================================================================
\section{附加图表列表}
%==============================================================================

除上述核心图表外,项目还生成了以下辅助图表:

\subsection{Phase 1: 特征工程图表}
\begin{itemize}
    \item \texttt{pbi\_by\_industry.png} - 按行业分类的PBI分布
    \item \texttt{pbi\_trend.png} - PBI随赛季的趋势变化
    \item \texttt{star\_makers.png} - Star Maker舞伴识别图
\end{itemize}

\subsection{Phase 2: 贝叶斯推断图表}
\begin{itemize}
    \item \texttt{ci\_width\_by\_week.png} - CI宽度按周次变化
    \item \texttt{ci\_width\_distribution.png} - CI宽度分布直方图
    \item \texttt{uncertainty\_by\_season.png} - 按赛季的不确定性
\end{itemize}

\subsection{Phase 3: 模拟器图表}
\begin{itemize}
    \item \texttt{ffi\_comparison.png} - FFI对比图
    \item \texttt{jfi\_comparison.png} - JFI对比图
    \item \texttt{weekly\_differences.png} - 每周差异分布
    \item \texttt{era\_analysis.png} - 按时代的分析
\end{itemize}

\subsection{Phase 4: Pareto优化图表}
\begin{itemize}
    \item \texttt{weight\_vs\_objectives.png} - 权重与目标的关系
    \item \texttt{tradeoff\_ratio.png} - 权衡比率图
\end{itemize}

\subsection{Phase 5: 推荐规则图表}
\begin{itemize}
    \item \texttt{dynamic\_weights.png} - 动态权重变化曲线
    \item \texttt{method\_comparison.png} - 方法对比汇总
    \item \texttt{summary\_infographic.png} - 综合信息图
\end{itemize}

%==============================================================================
\section{公式-图表-结论对应关系总结}
%==============================================================================

\begin{longtable}{|c|p{4cm}|p{4cm}|p{4.5cm}|}
\hline
\textbf{阶段} & \textbf{核心公式} & \textbf{对应图表} & \textbf{主要结论} \\
\hline
\endfirsthead
\hline
\textbf{阶段} & \textbf{核心公式} & \textbf{对应图表} & \textbf{主要结论} \\
\hline
\endhead
Phase 1 & 公式1(J\%)、公式2(PBI)、公式16(Divergence) & 全局扫描热力图、PBI分析图 & 社交媒体时代分歧加剧,改革必要 \\
\hline
Phase 2 & 公式5(约束)、公式6($P_w$)、公式7(CIW) & 贝叶斯汇总图、确定性一致性图 & 模型95.6\%准确,输出可信 \\
\hline
Phase 3 & 公式3(Rank)、公式4(Pct)、公式14(弹性) & 方法比较图、弹性图、案例图 & Pct更偏粉丝,Rank更公平 \\
\hline
Phase 4 & 公式8(J)、公式9(F)、公式10-13 & Pareto前沿图、方差分解图 & Rank有明确最优点,推荐50-50+Save \\
\hline
Phase 5 & 公式15(动态权重) & 最终推荐图 & 动态权重+Judges' Save,减少60-70\%争议 \\
\hline
\end{longtable}

%==============================================================================
\section{关键数值结果汇总}
%==============================================================================

\subsection{模型性能}
\begin{itemize}
    \item 淘汰预测精确匹配率:\textbf{95.6\%}
    \item Jaccard相似度:\textbf{0.960}
    \item F1分数:\textbf{0.963}
    \item 平均CI宽度:\textbf{0.288}
\end{itemize}

\subsection{方法对比(34季平均)}

\begin{center}
\begin{tabular}{|l|c|c|c|}
\hline
\textbf{指标} & \textbf{Rank方法} & \textbf{Percentage方法} & \textbf{优胜} \\
\hline
FFI (粉丝偏好) & 0.7670 & 0.7882 & Pct \\
\hline
JFI (评委偏好) & 0.7274 & 0.3735 & \textbf{Rank} \\
\hline
Fan-Elasticity & 0.1374 & 0.1216 & Rank \\
\hline
\end{tabular}
\end{center}

\subsection{方差分解}

\begin{center}
\begin{tabular}{|l|c|c|}
\hline
\textbf{因素} & \textbf{对Judge Score} & \textbf{对Fan Vote} \\
\hline
Celebrity & 52.6\% & 42.2\% \\
\hline
Pro Dancer & 28.6\% & 30.6\% \\
\hline
Season & 3.4\% & 9.3\% \\
\hline
Residual & 15.4\% & 17.9\% \\
\hline
\end{tabular}
\end{center}

\subsection{Star Makers(前5位)}
\begin{enumerate}
    \item Derek Hough (+8.1 J-lift, +1.76 F-lift)
    \item Mark Ballas (+5.9 J-lift, +1.17 F-lift)
    \item Valentin Chmerkovskiy (+5.9 J-lift, +0.16 F-lift)
    \item Julianne Hough (+3.8 J-lift, +2.02 F-lift)
    \item Maksim Chmerkovskiy (+3.4 J-lift, +0.99 F-lift)
\end{enumerate}

%==============================================================================
\section{最终推荐}
%==============================================================================

\subsection{推荐规则}

\begin{center}
\fbox{
\parbox{12cm}{
\centering
\textbf{推荐评分公式}\\[0.5em]
$Score(i,w) = \alpha(w) \cdot J\% + (1-\alpha(w)) \cdot \log(1 + F\%)$\\[0.5em]
其中 $\alpha(w)$: 50\% $\rightarrow$ 70\% 随周次递增\\[1em]
\textbf{配套机制}\\[0.5em]
Judges' Save: 对Bottom 2中评委分较高者进行拯救
}
}
\end{center}

\subsection{预期效果}

\begin{itemize}
    \item \textbf{公平性提升}:J从0.445提升至0.665(+49\%)
    \item \textbf{参与度保持}:F从0.691提升至0.704(+1.9\%)
    \item \textbf{争议减少}:极端事件预期减少60-70\%
    \item \textbf{历史验证}:Bobby Bones、Bristol Palin等案例将被阻止
\end{itemize}

\vfill
\begin{center}
\rule{10cm}{0.4pt}\\[0.5em]
\textit{MCM 2026 Problem C - Fairness-Engagement Equilibrium Model (FEEM)}\\
\textit{符号、公式与图表完整文档}\\
2026年2月
\end{center}

\end{document}

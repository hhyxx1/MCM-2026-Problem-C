\subsection{Data Pre-processing}

Before building the Fairness-Engagement Equilibrium Model (FEEM), a thorough examination of the COMAP-provided dataset \texttt{2026\_MCM\_Problem\_C\_Data.csv} was necessary. The dataset contains 423 contestants across 34 seasons of \textit{Dancing with the Stars} (DWTS), with judge scores recorded for up to 11 weeks per season. During our inspection, we identified several data quality issues that required systematic cleaning.

\subsubsection{Identified Data Anomalies}

\begin{enumerate}
    \item \textbf{Inconsistent Scoring Systems:} Different seasons employed different judging panels—earlier seasons (S1–S17) used a 3-judge system with a maximum of 30 points, while later seasons (S18–S34) used a 4-judge system with a maximum of 40 points. Direct comparison of raw scores would introduce systematic bias.
    
    \item \textbf{N/A and Missing Values:} The \texttt{judge4\_score} columns contain ``N/A'' strings for seasons with only 3 judges. Additionally, contestants who were eliminated or withdrew have scores recorded as 0 or ``N/A'' for subsequent weeks.
    
    \item \textbf{Multi-Dance Weeks:} In finale and special theme weeks, contestants perform multiple dances, resulting in individual judge scores exceeding 10 (accumulated scores). These values should not be treated as outliers but require proper normalization.
    
    \item \textbf{Withdrawal Cases:} Several contestants withdrew mid-season due to injury or personal reasons (e.g., Sara Evans in Season 3), leaving incomplete records that could skew weekly averages.
\end{enumerate}

\subsubsection{Data Cleaning Process}

To address these issues, we implemented the following cleaning procedures:

\begin{table}[h]
\centering
\caption{Data Cleaning Operations}
\begin{tabular}{|l|l|l|}
\hline
\textbf{Issue} & \textbf{Original Data} & \textbf{Cleaned Data} \\
\hline
Scoring system & 30-pt or 40-pt raw scores & Unified $J\%$ (0--100 scale) \\
\hline
N/A values (judge4) & ``N/A'' strings & Excluded from calculation \\
\hline
Post-elimination scores & 0 or ``N/A'' & Marked as missing (NaN) \\
\hline
Multi-dance weeks & Scores $>$ 10 per judge & Normalized by implied maximum \\
\hline
Withdrawals & Incomplete records & Flagged as ``withdrew'' status \\
\hline
\end{tabular}
\end{table}

\subsubsection{Judge Score Standardization ($J\%$)}

To enable cross-season comparison, we standardized all judge scores into a percentage scale using the following formula:

\begin{equation}
J\%_{i,w} = \frac{\sum_{j=1}^{n} S_{i,w,j}}{\text{MaxScore}_w} \times 100
\end{equation}

where $S_{i,w,j}$ is the score given by judge $j$ to contestant $i$ in week $w$, $n$ is the number of active judges (3 or 4), and $\text{MaxScore}_w$ is calculated as:

\begin{equation}
\text{MaxScore}_w = 
\begin{cases}
n \times 10 & \text{if } \max_j(S_{i,w,j}) \leq 10 \text{ (single dance)} \\
n \times 10 \times \lceil \frac{\max_j(S_{i,w,j})}{10} \rceil & \text{if } \max_j(S_{i,w,j}) > 10 \text{ (multi-dance)}
\end{cases}
\end{equation}

\subsubsection{Panel Data Conversion}

The original wide-format data (one row per contestant) was transformed into a long-format panel data structure with observation unit $(i, w)$—contestant $i$ in week $w$. This transformation is essential for:

\begin{itemize}
    \item Mixed-effects regression models with season and week fixed effects
    \item Bayesian inference of weekly fan vote shares
    \item Time-series analysis of contestant performance trajectories
\end{itemize}

\subsubsection{Covariate Standardization}

Celebrity features were standardized for regression analysis:

\begin{itemize}
    \item \textbf{Industry:} Grouped into 8 standardized categories (Actor, Musician, Athlete, Model, TV\_Personality, Reality\_Star, Politician, Other)
    \item \textbf{Region:} Binary indicator (US = 1, Non-US = 0)
    \item \textbf{Age:} Binned into 5 groups (18--25, 26--35, 36--45, 46--55, 55+)
\end{itemize}

\subsubsection{Cleaning Results Summary}

\begin{table}[h]
\centering
\caption{Data Cleaning Summary Statistics}
\begin{tabular}{|l|r|}
\hline
\textbf{Metric} & \textbf{Value} \\
\hline
Total contestants (raw) & 423 \\
Total contestants (valid) & 423 \\
Total (contestant, week) observations & 2,777 \\
Seasons covered & S1 -- S34 \\
Weeks per season (max) & 11 \\
Withdrawal cases excluded & 6 \\
Multi-dance week records normalized & 147 \\
\hline
\end{tabular}
\end{table}

After cleaning, the processed data was exported to three files:
\begin{itemize}
    \item \texttt{clean\_judge\_scores\_wide.csv} — Wide format with $J\%$ per week
    \item \texttt{clean\_judge\_scores\_long.csv} — Long format panel data
    \item \texttt{season\_summary.csv} — Season-level metadata
\end{itemize}

The normalized $J\%$ scores enable fair comparison across all 34 seasons, forming the foundation for subsequent Bayesian inference and Pareto optimization analyses.

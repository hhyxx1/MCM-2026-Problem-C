\documentclass[11pt]{article}
\usepackage[margin=1in]{geometry}
\usepackage{booktabs}
\usepackage{enumitem}

\pagestyle{empty}

\begin{document}

\begin{center}
{\LARGE \textbf{Memorandum}}\\[0.5cm]
\end{center}

\noindent\textbf{TO:} DWTS Executive Producers\\
\textbf{FROM:} MCM Analytics Team\\
\textbf{DATE:} January 2026\\
\textbf{RE:} Voting System Reform Recommendations\\

\hrule
\vspace{0.5cm}

\section*{Executive Summary}

Our analysis of 34 seasons of \textit{Dancing with the Stars} reveals that while the current voting system produces consistent outcomes \textbf{95.6\%} of the time, \textbf{2 documented instances} of fan voting overriding judges in Top-3 placements have generated significant controversy (notably Bobby Bones in Season 27).

We recommend a \textbf{Dynamic Log-Weighting} formula combined with a \textbf{Judges' Save} mechanism to reduce such controversies by an estimated \textbf{60--70\%} while maintaining fan engagement.

\section*{Key Findings}

\begin{enumerate}[leftmargin=*]
    \item \textbf{Current System Performance:}
    \begin{itemize}
        \item Judge-Favor Index (JFI): 0.727 --- outcomes moderately align with judge rankings
        \item Fan-Favor Index (FFI): 0.788 --- fan preferences have significant influence
        \item Prediction accuracy of our model: 95.6\%
    \end{itemize}
    
    \item \textbf{Identified Anomalies:}
    \begin{itemize}
        \item Bobby Bones (S27): Won despite consistently lower judge scores
        \item Bristol Palin (S11): Reached Top 3 with bottom-tier dancing
        \item Both cases involved organized fan voting campaigns
    \end{itemize}
    
    \item \textbf{Professional Partner Impact:}
    \begin{itemize}
        \item Pro dancers explain 38\% of judge score variance
        \item ``Star Makers'' like Derek Hough provide +8.1 point average lift
        \item This is a feature, not a bug---pairing matters for entertainment
    \end{itemize}
\end{enumerate}

\section*{Recommended Changes}

\subsection*{1. Dynamic Log-Weighting Formula}

$$Score = \alpha(w) \times J\% + (1-\alpha(w)) \times \log(1 + F\%)$$

\begin{itemize}
    \item \textbf{Weeks 1--3:} $\alpha = 50\%$ (equal weight, build audience)
    \item \textbf{Weeks 4--7:} $\alpha$ increases 5\% per week
    \item \textbf{Weeks 8+:} $\alpha = 70\%$ (merit-focused finale)
\end{itemize}

The logarithmic transformation dampens extreme fan vote advantages while still rewarding popularity.

\subsection*{2. Judges' Save Mechanism}

When two contestants are in the bottom:
\begin{enumerate}
    \item Both perform a final ``dance-off''
    \item Judges collectively save one based on cumulative performance
    \item Prevents worst-performing contestants from advancing
\end{enumerate}

\section*{Risk Warnings}

\begin{enumerate}[leftmargin=*]
    \item \textbf{Fan Perception:} Some viewers may feel their votes matter less. \textit{Mitigation:} Market as ``rewarding consistent improvement'' rather than ``reducing fan power.''
    
    \item \textbf{Judges' Save Controversy:} Judges may face backlash for ``subjective'' saves. \textit{Mitigation:} Use transparent cumulative score criteria.
    
    \item \textbf{Reduced Drama:} Fewer upsets may reduce watercooler moments. \textit{Mitigation:} The dance-off itself creates new dramatic tension.
\end{enumerate}

\section*{Verifiable Predictions}

Under the proposed rules, historical replay shows:
\begin{itemize}
    \item Bobby Bones (S27): Would be eliminated Week 6, not win
    \item Bristol Palin (S11): Would be eliminated Week 9, not Top 3
    \item Extreme events reduced from 2 to an estimated 0--1 per decade
\end{itemize}

\section*{Implementation}

\begin{itemize}
    \item No changes needed to voting infrastructure
    \item Judges' Save can be announced as a ``dramatic twist''
    \item Dynamic weighting can be disclosed or kept internal
    \item \textbf{Recommend:} Pilot test in one season before full rollout
\end{itemize}

\vfill
\hrule
\vspace{0.2cm}
\noindent\textit{This analysis is based on 34 seasons, 421 contestants, and 2,777 weekly observations. Full methodology available in accompanying technical report.}

\end{document}
